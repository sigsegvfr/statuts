\documentclass[a4paper,oneside,10pt]{article}

\usepackage[french]{babel}
\usepackage[utf8]{inputenc}
\usepackage[T1]{fontenc}
\usepackage{gensymb}
\usepackage{color}
\usepackage[colorlinks=false,
  hidelinks,
  pdfusetitle
  ]{hyperref}
\usepackage[autostyle]{csquotes}
\usepackage{vmargin}
\usepackage{fancyhdr}

\setlength{\parskip}{0.5cm}

\begin{document}

\author{Association sigsegv}
\date{}
\title{SIGSEGV - Procès verbal pour l'ouverture de compte}

{\centering {\textbf {\Huge SIGSEGV}} \\
}
\vspace{5mm}
{\centering {\textbf {\Huge -}}\\
}
\vspace{5mm}
{\centering {\textbf {EXTRAIT DU PROCES-VERBAL DU BUREAU DIRECTEUR DE L'ASSOCIATION EN DATE DU 23 août 2017}}\\
}

\section*{}

Le 23 août 2017, à 18 heures, le bureau directeur de l'association sigsegv s'est réuni à Grenoble sous la présidence de Monsieur Charles-Henri Bruyand.

Membres présents :

\begin{itemize}
  \item M. Charles-Henri Bruyant - Président
  \item M. Rémi Gacogne - Trésorier
  \item M. Clément Larrivé - Secrétaire
  \item M. Guillaume de Lafond - Vice Président
\end{itemize}

Le quorum étant atteint, le bureau directeur peut valablement délibérer.

Il donne pouvoir à M. Gacogne et M. de Lafond, respectivement trésorier et vice-président:\\
\begin{itemize}
  \item pour ouvrir un compte bancaire dans la banque qu'ils estimeront la plus intéressante pour l'association ;
  \item pour faire toutes les opérations concernant le fonctionnement du compte, notamment de signer tous ordres,
    reçus, chèques, virements, et faire tous versements et tous retraits,
    de retirer ou de verser toutes pièces comptables, de donner toutes quittances et décharges, et, de façon
    générale, effectuer toutes opérations pour le compte de la collectivité sus-dénommée.\\
\end{itemize}

M. Gacogne et M. de Lafond pourront agir séparément, mais pour la bonne marche de l'association, toutes ces opérations
seront assurées par M. Gacogne, en sa qualité de trésorier. M. de Lafond n'utilisera son pouvoir de signature de
documents bancaires qu'en cas d'absence prolongée du trésorier qui pourrait nuire au fonctionnement de
l'association.


Le présent pouvoir est valable jusqu'à révocation expresse de notre part. Il sera obligatoirement renouvelé
de notre chef, au cas où un changement interviendrait au sein de la collectivité.

À Grenoble, le 23 août 2017

Pour copie certifiée conforme.

Le Président

Un administrateur

\end{document}
